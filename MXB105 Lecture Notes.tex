\documentclass{article}
\usepackage{template}

\usepackage{chngcntr} % Reset counter within sections
\usepackage{multicol}

% Create nice frames
\usepackage{mdframed}
\mdfsetup{skipabove=\topskip,skipbelow=\topskip}
\mdfdefinestyle{exampledefault}{%
	rightline=true,
	innerleftmargin=10,innerrightmargin=10,
	leftmargin=20,rightmargin=20,
	topline=false,bottomline=false,
	frametitlefont=\bf\large
}
\mdfdefinestyle{exampledefaultcols}{%
	rightline=false,leftline=true,
	innerleftmargin=5,innerrightmargin=5,
	leftmargin=0,rightmargin=0,
	topline=false,bottomline=false,
	frametitlefont=\bf\large
}

\counterwithin*{equation}{section}
\counterwithin*{equation}{subsection}

\pagestyle{fancy}
\setlength\headheight{24pt}

\lhead{\className}
\rhead{\leftmark}
\cfoot{\thepage}

\newcommand{\uniTitle}{Queensland University of Technology}
\newcommand{\className}{Calculus and Differential Equations}
\newcommand{\classTime}{2021, Semester 2}
\newcommand{\classInstructorName}{Dr Vivien Challis}
\newcommand{\authorName}{Tarang Janawalkar}
\newcommand{\authorStudentNumber}{n11032201}
\newcommand{\classCode}{MXB105}

\usepackage[
    type={CC},
    modifier={by-nc-sa},
    version={4.0},
    imagewidth={5em},
]{doclicense}

\date{}

\begin{document}

\begin{titlepage}
    \vspace*{\fill}
    \begin{center}
        \LARGE
        \textbf{\className}
        \texorpdfstring{\\}{ }
        \uniTitle
        \texorpdfstring{\\}{ }
        \texorpdfstring{\vspace{0.3in}}{ }
        \normalsize\textit{\classInstructorName}
        \texorpdfstring{\\}{ }
        \classTime
    \end{center}
    \begin{center}
        \textbf{\authorName}
    \end{center}
    \vspace*{\fill}
    \doclicenseThis
    \thispagestyle{empty}
\end{titlepage}
\newpage

\tableofcontents
\newpage

\section{Integration Techniques}
\subsection{Derivative Table}
Let $f(x)$ be a function, and $a\in\mathbb{R}$ be a constant.
\begin{table}[H]
    \renewcommand*{\arraystretch}{1.5}
    \centering
    \begin{tabular}{>{$}c<{$} | >{$}c<{$}}
        \toprule
            f & \dv{f}{x} \\
        \midrule
            x^a & a x^{a-1} \\
            \sqrt{x} & \displaystyle \frac{1}{2\sqrt{x}} \\
            a^x & \ln{\left( a \right)} a^x \\
            \e^x & \e^x \\
            \log_a{x}, \: a\in \mathbb{R}\backslash\left\{ 0 \right\} & \displaystyle \frac{1}{a\ln{x}} \\
            \ln{x} & \displaystyle \frac{1}{x} \\
        \bottomrule
    \end{tabular}
    \begin{tabular}{>{$}c<{$} | >{$}c<{$}}
        \toprule
            f & \dv{f}{x} \\
        \midrule
            a & 0 \\
            x & 1 \\
            a_1 u(x) \pm a_2 v(x) & \displaystyle a_1\dv{u}{x} \pm a_2\dv{v}{x} \\
            u(x)v(x) & \displaystyle \dv{u}{x}v + u\dv{v}{x} \\
            \displaystyle \frac{u(x)}{v(x)} & \displaystyle \frac{\dv{u}{x}v - u\dv{v}{x}}{{v(x)}^2} \\
            u\bigl( v\left( x \right) \bigr) & \displaystyle \dv{u}{v}\dv{v}{x} \\
        \bottomrule
    \end{tabular}
\end{table}
\begin{table}[H]
    \renewcommand*{\arraystretch}{1.5}
    \centering
    \begin{tabular}{>{$}c<{$} | >{$}c<{$}}
        \toprule
            f & \dv{f}{x} \\
        \midrule
            \sin{\left( ax \right)} &  a\cos{\left( ax \right)} \\
            \cos{\left( ax \right)} & -a\sin{\left( ax \right)} \\
            \tan{\left( ax \right)} &  a\sec^2{\left( ax \right)} \\
            \cot{\left( ax \right)} & -a\csc^2{\left( ax \right)} \\
            \sec{\left( ax \right)} &  a\sec{\left( ax \right)}\tan{\left( ax \right)} \\
            \csc{\left( ax \right)} & -a\csc{\left( ax \right)}\cot{\left( ax \right)} \\
            \arcsin{\left( ax \right)} & \displaystyle  \frac{a}{\sqrt{1-a^2x^2}} \\
            \arccos{\left( ax \right)} & \displaystyle -\frac{a}{\sqrt{1-a^2x^2}} \\
            \arctan{\left( ax \right)} & \displaystyle  \frac{a}{1+a^2x^2} \\
            \arccot{\left( ax \right)} & \displaystyle -\frac{a}{1+a^2x^2} \\
            \arcsec{\left( ax \right)} & \displaystyle  \frac{1}{x\sqrt{a^2x^2 - 1}} \\
            \arccsc{\left( ax \right)} & \displaystyle -\frac{1}{x\sqrt{a^2x^2 - 1}} \\
        \bottomrule
    \end{tabular}
    \begin{tabular}{>{$}c<{$} | >{$}c<{$}}
        \toprule
            f & \dv{f}{x} \\
        \midrule
            \sinh{\left( ax \right)} &  a\cosh{\left( ax \right)} \\
            \cosh{\left( ax \right)} &  a\sinh{\left( ax \right)} \\
            \tanh{\left( ax \right)} &  a\sech^2{\left( ax \right)} \\
            \coth{\left( ax \right)} & -a\csch^2{\left( ax \right)} \\
            \sech{\left( ax \right)} & -a\sech{\left( ax \right)}\tan{\left( ax \right)} \\
            \csch{\left( ax \right)} & -a\csch{\left( ax \right)}\cot{\left( ax \right)} \\
            \arcsinh{\left( ax \right)} & \displaystyle  \frac{a}{\sqrt{1+a^2x^2}} \\
            \arccosh{\left( ax \right)} & \displaystyle  \frac{a}{\sqrt{1-a^2x^2}} \\
            \arctanh{\left( ax \right)} & \displaystyle  \frac{a}{1-a^2x^2} \\
            \arccoth{\left( ax \right)} & \displaystyle  \frac{a}{1-a^2x^2} \\
            \arcsech{\left( ax \right)} & \displaystyle -\frac{1}{a\left( 1+ax \right)\sqrt{\frac{1-ax}{1+ax}}} \\
            \arccsch{\left( ax \right)} & \displaystyle -\frac{1}{ax^2\sqrt{1+\frac{1}{a^2x^2}}} \\
        \bottomrule
    \end{tabular}
    \caption{Derivatives of Elementary Functions}
\end{table}
\subsection{Partial Fractions}
\begin{definition}[Partial Fraction Decomposition]
    \textbf{Partial fraction decomposition} is a \linebreak method where a rational function $\frac{P(x)}{Q(x)}$ is rewritten as a sum of fraction.
\end{definition}
\begin{table}[H]
    \renewcommand*{\arraystretch}{1.5}
    \centering
    \begin{tabular}{c | c}
        \toprule
            Factor in denominator & Term in partial fraction decomposition \\
        \midrule
            $ax+b$ & $\frac{A}{ax+b}$ \\
            $\left(ax+b\right)^k$ & $\frac{A_1}{ax+b} + \frac{A_2}{\left( ax+b \right)^2} + \cdots + \frac{A_k}{\left( ax+b \right)^k}, \: k \in \mathbb{N}$ \\
            $ax^2+bx+c$ & $\frac{A}{ax^2+bx+c}$ \\
            $\left(ax^2+bx+c\right)^k$ & $\frac{A_1x+B_1}{ax^2+bx+c} + \frac{A_2}{\left( ax+b \right)^2} + \cdots + \frac{A_k}{\left( ax+b \right)^k}, \: k \in \mathbb{N}$ \\
        \bottomrule
    \end{tabular}
    \caption{Partial Fraction Forms}
\end{table}
\subsection{Integration by Parts}
\begin{theorem}
\begin{equation*}
    \int u \dd{v} = uv - \int v \dd{u}
\end{equation*}
\end{theorem}
\begin{proof}
    \begin{align*}
        \dv{}{x}\left( u(x)v(x) \right) &= \dv{u(x)}{x}v(x) + u(x)\dv{v(x)}{x} \\
        u(x)\dv{v(x)}{x} &= \dv{}{x}\left( u(x)v(x) \right) - \dv{u(x)}{x}v(x) \\
        \int u(x)\dv{v(x)}{x} \dd{x} &= \int \dv{}{x}\left( u(x)v(x) \right) \dd{x} - \int \dv{u(x)}{x}v(x) \dd{x} \\
        \int u(x)\dd{v(x)} &= u(x)v(x) - \int v(x) \dd{u(x)}
    \end{align*}
\end{proof}
\subsection{Integration by Substitution}
\begin{theorem}
    \begin{equation*}
        \int f\bigl(g\left( x \right)\bigr)\dv{g(x)}{x} \dd{x} = \int f(u) \dd{u}, \: \text{where } u = g(x)
    \end{equation*}
\end{theorem}
\subsection{Trigonometric Substitutions}
\begin{table}[H]
    \renewcommand*{\arraystretch}{1.5}
    \centering
    \begin{tabular}{>{$}c<{$} | >{$}c<{$} >{$}c<{$} | >{$}c<{$}}
        \toprule
        \text{Form} & \text{Substitution} & \text{Result} & \text{Domain} \\ 
        \midrule
        \left(a^2-b^2x^2\right)^n & \displaystyle x=\frac{a}{b}\sin{\left( \theta \right)} & a^2\cos^2{\left( \theta \right)} & \theta\in \left[ -\frac{\pi}{2},\: \frac{\pi}{2} \right] \\
        \left(a^2+b^2x^2\right)^n & \displaystyle x=\frac{a}{b}\tan{\left( \theta \right)} & a^2\sec^2{\left( \theta \right)} & \theta\in \left( -\frac{\pi}{2},\: \frac{\pi}{2} \right) \\
        \left(b^2x^2-a^2\right)^n & \displaystyle x=\frac{a}{b}\sec{\left( \theta \right)} & a^2\tan^2{\left( \theta \right)} & \theta\in \left[ 0,\: \frac{\pi}{2} \right) \cup \left(\frac{\pi}{2},\: \pi\right] \\
        \bottomrule
    \end{tabular}
    \caption{Trigonometric substitutions for various forms.}
\end{table}
\newpage
\section{Limits, Continuity and Differentiability}
\subsection{Limits}
\begin{theorem}[Limits]
    $\displaystyle\lim_{x\to x_0} f(x)$ exists if and only if
    $\displaystyle\lim_{x\to {x_0}^+} f(x)$ and $\displaystyle\lim_{x\to {x_0}^-} f(x)$
    exist and are equal.
    
    \noindent For $f:S\to T$,
    \begin{equation*}
        I\subseteq S:\exists L\in I:\lim_{x\to x_0} f(x) = L \iff \lim_{x\to x_0^+} f(x) = \lim_{x\to x_0^-} f(x) = L
    \end{equation*}
\end{theorem}
\begin{theorem}[L'H\^opital's Rule]
    For two differentiable functions $f(x)$ and $g(x)$.
    If $\displaystyle \lim_{x\to x_0}f(x)=\lim_{x\to x_0}g(x)=0$,
    or $\displaystyle \lim_{x\to x_0}f(x)=\displaystyle \lim_{x\to x_0}g(x)=\pm\infty$,
    then
    $\lim_{x\to x_0}\frac{f(x)}{g(x)} = \lim_{x\to x_0}\frac{f'(x)}{g'(x)}$
    (as long as the limit exists, or diverges to $\pm\infty$).
\end{theorem}
\subsection{Continuity}
\begin{theorem}[Continuity at a Point]
    $f(x)$ is continuous at $c$ iff $\displaystyle \lim_{x\to c} f(x) = f(c)$.
\end{theorem}
\begin{theorem}[Continuity over an Interval]
    $f(x)$ is continuous on $I$ if $f(x)$ is continuous for all $x\in I$.
\begin{itemize}
    \item $f(x)$ is continuous on $I:\left( a,\:b \right)$ if it is continuous for all $x\in I$.
    \item $f(x)$ is continuous on $I:\left[ a,\:b \right]$ if it is continuous for all $x\in I$, but only right continuous at $a$ and left continuous at $b$.
\end{itemize}
    If $f(x)$  is continuous on $\left(-\infty,\:\infty\right)$, $f(x)$ is continuous everywhere.
\end{theorem}
\begin{theorem}[Intermediate Value Theorem]
    If $f(x)$ is continuous on $I:\left[ a, \: b \right]$ and $c$ is any number between $f(a)$ and $f(b)$, inclusive, then there exists an $x\in I$ such that $f(x)=c$.
\end{theorem}
\subsection{Differentiability}
\begin{theorem}[Differentiability]
    $f(x)$ is differentiable at $x=x_0$ iff
    \begin{equation*}
        \lim_{x\to x_0} \frac{f(x)-f(x_0)}{x-x_0}
    \end{equation*}
    exists. When this limit exists, it defines the derivative
    \begin{equation*}
        \dv{f(x)}{x}\vert_{x=x_0} = \lim_{h\to 0} \frac{f(x_0+h)-f(x_0)}{h}
    \end{equation*}
\end{theorem}
\begin{theorem}
    $f(x)$ is differentiable on $I$ if $f(x)$ is differentiable for all $x_0\in I$.
\end{theorem}
\begin{theorem}
    Differentiability implies continuity.
\end{theorem}
\begin{theorem}[Mean Value Theorem]
    If $f(x)$ is continuous on $I:\left[ a,\:b \right]$ and differentiable on $I$, then there exists a point $x_0\in I$ such that
    \begin{equation*}
        \dv{f}{x}(x_0)=\frac{f(b)-f(a)}{b-a}
    \end{equation*}
\end{theorem}
\newpage
\section{Definite Integrals}
\begin{theorem} 
    If $f(x)$ is continuous on an interval $I:\left[ a,\:b \right]$, then the net signed area $A$ between the graph of $f(x)$ and the interval $I$ is
    \begin{equation*}
        A = \int_a^b f(x) \dd{x} 
    \end{equation*} 
\end{theorem}
\begin{figure}[H]
	\begin{mdframed}[style=exampledefault,frametitle={Properties of Definite Integrals}]
		\begin{theorem}
			Suppose that $f(x)$ and $g(x)$ are continuous on the interval $I$, with $a,\:b,\:c\in I$ and $k\in\mathbb{R}$ then
			\begin{enumerate}[label=\normalfont\alph*)]
				\item $\displaystyle\int_a^a f(x) \dd{x} = 0$.
				\item $\displaystyle\int_a^b f(x) \dd{x} = -\int_b^a f(x) \dd{x}$.
				\item $\displaystyle\int_a^b kf(x) \dd{x} = k\int_a^b f(x) \dd{x}$.
				\item $\displaystyle\int_a^b \bigl(f(x) \pm g(x)\bigr) \dd{x} = \int_a^b f(x) \dd{x} \pm \int_a^b g(x) \dd{x}$.
				\item $\displaystyle\int_a^b f(x) \dd{x} = \int_a^c f(x) \dd{x} + \int_c^b f(x) \dd{x}$.
			\end{enumerate}
		\end{theorem}
	\end{mdframed}
\end{figure}
\subsection{Riemann Sums}
\begin{theorem}
    Let $A$ be the area under $f(x)$ on the interval $\left[ a,\:b \right]$, then
    \begin{equation*}
        \int_a^b f(x) \dd{x} = \lim_{\max{\Delta x_k}\to 0} \sum_{k=1}^n f(x_k) \Delta x_k
    \end{equation*}
    where $n$ is the number of rectangles, $x_k$ is the centre of the rectangle $k$, and $\Delta x_k$ is the width of the rectangle $k$. If every rectangle has the same width, then
    \begin{equation*}
        \forall k:\Delta x_k = \frac{b-a}{n}
    \end{equation*}
\end{theorem}
\subsection{Fundamental Theorem of Calculus}
The fundamental theorem of calculus provides a logical connection between infinite series (definite integrals) and antiderivatives (indefinite integrals).
\begin{theorem}[The Fundamental Theorem of Calculus: Part 1]
    If $f(x)$ is continuous on $\left[ a,\:b \right]$ and $F$ is any antiderivative of $f$ on $\left[ a,\:b \right]$ then
    \begin{equation*}
        \int_a^b f(x)\dd{x} = F(b) - F(a)
    \end{equation*}
    Equivalently
    \begin{equation*}
        \int_a^b \dv{x}F(x) \dd{x} = F(b) - F(a) \equiv \left.F(x)\right|_a^b
    \end{equation*}
\end{theorem}
\begin{theorem}[The Fundamental Theorem of Calculus: Part 2]
    If $f(x)$ is continuous on $I$ then it has an antiderivative on $I$. In particular, if $a\in I$, then the function $F$ defined by
    \begin{equation*}
        F(x) = \int_a^x f(t)\dd{t}
    \end{equation*}
    is an antiderivative of $f(x)$. That is,
    \begin{equation*}
        \dv{x}F(x) = f(x) \iff \dv{}{x}(\int_a^x f(t) \dd{t}) = f(x)
    \end{equation*}
\end{theorem}
\begin{theorem}
    Differentiation and integration are inverse operations.
\end{theorem}
\newpage
\subsection{Taylor and Maclaurin Polynomials}
\begin{theorem}[Taylor Polynomials]
    If $f(x)$ is a $n$ differentiable function at $x_0$, then the $n$th degree Taylor polynomial for $f(x)$ near $x_0$, is given by
    \begin{equation*}
        f(x) \approx P_n(x) = \sum_{k=0}^n \frac{f^{\left( k \right)}(x_0)}{k!} \left( x-x_0 \right)^k
    \end{equation*}
\end{theorem}
\begin{theorem}[Maclaurin Polynomials]
    Evaluating a Taylor polynomial near $0$, gives the $n$th degree Maclaurin polynomial for $f(x)$
    \begin{equation*}
        f(x) \approx P_n(x) = \sum_{k=0}^n \frac{f^{\left( k \right)}(0)}{k!} x^k
    \end{equation*}
\end{theorem}
\begin{theorem}[Error in Approximation]
    Let $R_n(x)$ denote the difference between $f(x)$ and its $n$th Taylor polynomial, that is 
    \begin{equation*}
        R_n(x) = f(x) - P_n(x) = f(x) - \sum_{k=0}^n \frac{f^{\left( k \right)}(x_0)}{k!} \left( x-x_0 \right)^k
    \end{equation*}
\end{theorem}
\newpage
\section{Taylor and Maclaurin Series}
\subsection{Infinite Series}
\begin{definition}[Taylor Series]
    If $f(x)$  has derivatives of all orders at $x_0$, then the Taylor series for $f(x)$ about $x=x_0$ is given by
    \begin{equation*}
        f(x) = \sum_{n=0}^{\infty} \frac{f^{\left( n \right)}(x_0)}{n!}\left( x-x_0 \right)^n
    \end{equation*}
\end{definition}
\begin{definition}[Maclaurin Series]
    If a Taylor series is centred on $x_0=0$, it is called a Maclaurin series, defined by
    \begin{equation*}
        f(x) = \sum_{n=0}^{\infty} \frac{f^{\left( n \right)}(0)}{n!} x^n
    \end{equation*}
\end{definition}
\begin{definition}[Power Series]
    Both Taylor and Maclaurin series are examples of \textbf{power series} defined as follows
    \begin{equation*}
        \sum_{n=0}^{\infty} c_n\left( x-x_0 \right)^n
    \end{equation*}
\end{definition}
\subsection{Convergence Tests}
For any infinite series of the form $\displaystyle\sum_{i=i_0}^\infty a_i$.
\begin{mdframed}[style=exampledefaultcols,frametitle={Alternating Series}]
    \textbf{Conditions} $a_i = \left( -1 \right)^i b_i$ or $a_i = \left( -1 \right)^{i+1} b_i$. $b_i>0$.
    \begin{equation*}
        \text{Is $b_{i+1}\leqslant b_i$ \& $\lim_{i\to\infty}b_i=0$?}\:
        \begin{cases}
            \text{YES} & \text{$\sum a_i$ Converges} \\
            \text{NO} & \text{Inconclusive}
        \end{cases}
    \end{equation*}
\end{mdframed}
\begin{mdframed}[style=exampledefaultcols,frametitle={Ratio Test}]
    \textbf{Conditions} $\forall i:a_i>0$ and $\displaystyle \lim_{i\to\infty}\frac{a_{i+1}}{a_i}\neq 1$.
    \begin{equation*}
        \text{Is $\lim_{i\to\infty}\frac{a_{i+1}}{a_i} < 1$?}\:
        \begin{cases}
            \text{YES} & \text{$\sum a_i$ Converges} \\
            \text{NO} & \text{$\sum a_i$ Diverges}
        \end{cases}
    \end{equation*}
    \begin{equation*}
        \text{Is $\lim_{i\to\infty}\abs{\frac{a_{i+1}}{a_i}} < 1$?}\:
        \begin{cases}
            \text{YES} & \text{$\sum a_i$ Converges Absolutely} \\
            \text{NO} & \text{$\sum a_i$ Diverges}
        \end{cases}
    \end{equation*}
\end{mdframed}
\newpage
\section{Multivariable Calculus}
\begin{definition}
    A function is multivariable if its domain consists of several variables. In the reals, these functions are defined
    \begin{equation*}
        f:\mathbb{R}^n\to\mathbb{R}
    \end{equation*}
\end{definition}
\newpage
\section{Double and Triple Integrals}
\newpage
\section{Vector-Valued Functions}
\newpage
\section{First-Order Differential Equations}
\newpage
\section{Second-Order Differential Equations}
\newpage

\end{document}